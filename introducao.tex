\chapter{INTRODUÇÃO}
\label{INTRODUCAO}
\vspace*{-2cm}

Segundo \citeonline{raghupathi2014} a evolução das tecnologias pode agregar valor e benefícios para as várias áreas do conhecimento humano, incluindo a agricultura e permite extrair conhecimento a partir de associações e análise de grandes volumes de dados.

Estes dados ficam armazenadas em locais chamados bancos de dados. Segundo \citeonline{hkorth}, um banco de dados é um conjunto de dados que se relacionam entre si.

Com o acúmulo de dados, os Sistemas Gerenciadores de Banco de Dados (SGBD) podem ficar muito extensos e difíceis de analisar em um tempo aceitável, é necessário utilizar metodologias de “Big Data”. De acordo \citeonline{wkent}, o termo “Big Data” refere-se a um conjunto de metodologias que são destinadas a operar em bancos de dados tão volumosos nos quais os aplicativos de processamento de dados tradicionais ainda não conseguem processar em um tempo tolerável.

Porém, para que estes dados gerem informações de forma assertiva, é necessário garantir que todos os dados sejam coerentes, pois segundo \citeonline{lima2009}, a garantia de uma informação de qualidade é condição essencial para a análise objetiva.

\section{OBJETIVOS DO ESTÁGIO}

Estudar boas práticas para de qualidade dos dados para avaliação do(s) banco(s) de dados em projetos desenvolvidos pela empresa, buscando aplicar estes conceitos para criação de algoritmos visando à melhoria dos dados destes projetos.
