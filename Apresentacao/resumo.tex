%
%%%%%%%%%% Resumo %%%%%%%%%%
\newpage
\singlespacing\thispagestyle{empty}
\begin{center}
{\bf RESUMO}
\end{center}

Esta tese propôe um modelo para o tráfego de veículos em uma pista com duas faixas que considere o comportamento do motorista. Os principais modelos encontrados na literaturas apresentam em sua estrutura equações diferenciais ou integro-diferenciais que exigem muitas variáveis para calibração do modelo. Por isso, usa-se neste trabalho a modelagem computacional via autômatos celulares. Os autômatos celulares possuem uma estrutura bastante simples e são capazes de modelar diversos fenômenos complexos por meio de regras simples. Primeiramente, desenvolveu-se um modelo de tráfego em uma pista com uma faixa onde avalia-se a mudança do regime do tráfego por meio de um parâmetro estocástico. A segunda parte do trabalho, ainda não desenvolvida, será feito um modelo com duas faixas considerando diferentes comportamentos do motorista por uma distribuição de probabilidade contínua. Esse modelo será validado por meio de dados experimentais disponíveis na literatura e também por dados reais do tráfego.


\vspace{1.5cm}
\noindent Palavras-chave: simulação computacional, tráfego de veículos, autômatos celulares.



%
%%%%%%%%%% Abstract %%%%%%%%%%



